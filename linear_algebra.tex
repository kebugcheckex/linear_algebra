\documentclass[letterpaper,11pt]{article}
\usepackage{amsmath}
\usepackage{amssymb}
\usepackage{amsfonts}
\usepackage{enumerate}
\title{Linear Algebra Theorems}
\author{Linear Algebra and Its Applications (David C. Lay)}

\begin{document}
	\maketitle
	\section{Linear Equations in Linear Algebra}
		\subsection{Systems of Linear Equations}
			A system of linear equations has either
			\begin{enumerate}
				\item no solution, or
				\item exactly one solution, or
				\item infinitely many solutions.
			\end{enumerate}
		\subsection{Row Reduction and Echelon Forms}
			\subsubsection{Definition}
				A rectangular matrix is in \textbf{echelon form} (or \textbf{row echelon form}) if it has the following three properties:
				\begin{enumerate}
					\item All nonzero rows are above any rows of all zeros.
					\item Each leading entry of a row is in a column to the right of the leading entry of the row above it.
					\item All entries in a column below a leading entry are zeros.
				\end{enumerate}
				If a matrix in echelon form satisfies the following additional conditions, then it is in \textbf{reduced echelon form} (or \textbf{reduced row echelon form}).
				\begin{enumerate}
					\item The leading entry in each nonzero row is 1.
					\item Each leading 1 is the only nonzero entry in its column.
				\end{enumerate}
			\subsubsection{Theorem 1}
				\textbf{Uniqueness of the Reduced Echelon Form:} Each matrix is row equivalent to one and only one reduced echelon form.
			\subsubsection{Definition}
				A \textbf{pivot position} in a matrix $A$ is a location in $A$ that corresponds to a leading 1 in the reduced echelon form of $A$. A \textbf{pivot column} is a column of $A$ that contains a pivot position.
			\subsubsection{Theorem 2}
				\textbf{Existence and Uniqueness Theorem:} A linear system is consistent if and only if the rightmost column of the augmented matrix is \textit{not} a pivot position -- that is, if and only if an echelon form of the augmented matrix has \textit{no} row of the form
				\begin{equation}
					\begin{bmatrix}
						0 & \cdots & 0 & b
					\end{bmatrix}
					\textrm{with } b \textrm{ nonzero}
				\end{equation}
				If a linear system is consistent, then the solution set contains either (i) a unique solution, when there are no free variables, or (ii) infinitely many solutions, when there is at least one free variable.
		\subsection{Vector Equations}
			\textbf{Algebraic Properties of $\mathbb{R}^n$:} For all $\mathbf{u}$, $\mathbf{v}$, $\mathbf{w}$ in $\mathbb{R}^n$ and all scalar $c$ and $d$:
			\begin{enumerate}[(i)]
				\item $\mathbf{u}+\mathbf{v}=\mathbf{v}+\mathbf{u}$
				\item $(\mathbf{u}+\mathbf{v})+\mathbf{w}=\mathbf{u}+(\mathbf{v}+\mathbf{w})$
				\item $\mathbf{u}+\mathbf{0}=\mathbf{0}+\mathbf{u}=\mathbf{u}$
				\item $\mathbf{u}+(-\mathbf{u})=-\mathbf{u}+\mathbf{u}=\mathbf{u},\textrm{where } -\mathbf{u} \textrm{ denotes } (-1)\mathbf{u}$
				\item $c(\mathbf{u}+\mathbf{v})=c\mathbf{u}+c\mathbf{v}$
				\item $(c+d)\mathbf{u}=c\mathbf{u}+d\mathbf{u}$
				\item $c(d\mathbf{u})=(cd)(\mathbf{u})$
				\item $1\mathbf{u}=\mathbf{u}$
			\end{enumerate}
			
			A vector equation
			\begin{equation}
				x_1\mathbf{a}_1+x_2\mathbf{a}_2+\cdots+x_n\mathbf{a}_n=\mathbf{b}
			\end{equation}
			has the same solution set as the linear system whose augmented matrix is
			\begin{equation}
				\label{equ:aug}
				\begin{bmatrix}
					\mathbf{a}_1 & \mathbf{a}_2 & \cdots & \mathbf{a}_n & \mathbf{b}
				\end{bmatrix}
			\end{equation}
			In particular, $\mathbf{b}$ can be generated by a linear combination of $\mathbf{a}_1$, $\cdots$, $\mathbf{a}_n$ if and only if there exists a solution to the linear system corresponding to (\ref{equ:aug}).
			
			\subsubsection{Definition}
				If $\mathbf{v}_1$, $\cdots$, $\mathbf{v}_p$ are in $\mathbb{R}^n$, then the set of all linear combinations of $\mathbf{v}_1$,$\cdots$,$\mathbf{v}_p$ is denoted by Span \{$\mathbf{v}_1$,$\cdots$,$\mathbf{v}_p$\} and is called the \textbf{subset of $\mathbb{R}^n$ spanned} (or \textbf{generated}) \textbf{by}  $\mathbf{v}_1$, $\cdots$, $\mathbf{v}_p$. That is, Span \{$\mathbf{v}_1$, $\cdots$, $\mathbf{v}_p$\} is the collection of all vectors that can be written in the form
				\begin{equation}
					c_1\mathbf{v}_1+c_2\mathbf{v}_2+\cdots+c_p\mathbf{v}_p
				\end{equation}
				with $c_1$, $\cdots$, $c_p$ scalars.
		\subsection{The Matrix Equation $A\mathbf{x}=\mathbf{b}$}
			\subsubsection{Definition}
				If $A$ is an $m\times n$ matrix, with columns $\mathbf{a}_1$, $\cdots$, $\mathbf{a}_n$, and if $\mathbf{x}$ is in $\mathbb{R}^n$, then the product of $A$ and $\mathbf{x}$, denoted by $A\mathbf{x}$, is \textbf{the linear combination of the columns of $A$ using the corresponding entries in $\mathbf{x}$ as weights}; that is
				\begin{equation}
					A\mathbf{x}=
					\begin{bmatrix}
						\mathbf{a}_1 & \mathbf{a}_2 & \cdots & \mathbf{a}_n
					\end{bmatrix}
					\begin{bmatrix}
						x_1 \\ x_2 \\ \vdots \\ x_n
					\end{bmatrix}
					=x_1\mathbf{a}_1+x_2\mathbf{a}_2+\cdots+x_n\mathbf{a}_n
				\end{equation}
			\subsubsection{Theorem 3}
				If $A$ is an $m\times n$ matrix, with columns $\mathbf{a}_1$, $\cdots$, $\mathbf{a}_n$, and if $\mathbf{b}$ is in $\mathbb{R}^n$, the matrix equation
				\begin{equation}
					A\mathbf{x}=\mathbf{b}
				\end{equation}
				has the same solution set as the vector equation
				\begin{equation}
					x_1\mathbf{a}_1+x_2\mathbf{a}_2+\cdots+x_n\mathbf{a}_n=\mathbf{b}
				\end{equation}
				which, in turn, has the same solution set as the system of linear equations whose augmented matrix is
				\begin{equation}
					\begin{bmatrix}
						\mathbf{a}_1 & \mathbf{a}_2 & \cdots & \mathbf{a}_n & \mathbf{b}
					\end{bmatrix}
				\end{equation}
			
			The equation $A\mathbf{x}=\mathbf{b}$ has a solution if and only if $\mathbf{b}$ is a linear combination of the columns of $A$.
			
			\subsubsection{Theorem 4}
				Let $A$ be an $m\times n$ matrix. Then the following statements are logically equivalent. That is, for a particular $A$, either they are all true statements or they are all false.
				\begin{enumerate}[a.]
					\item For each $\mathbf{b}$ in $\mathbb{R}^m$, the equation $A\mathbf{x}=\mathbf{b}$ has a solution.
					\item Each $\mathbf{b}$ in $\mathbb{R}^m$ is a linear combination of the columns of $A$.
					\item The columns of $A$ span  $\mathbb{R}^m$.
					\item $A$ has a pivot position in every row.
				\end{enumerate}
			\subsubsection{Theorem 5}
				If $A$ is an $m\times n$ matrix, $\mathbf{u}$ and $\mathbf{v}$ are vectors in $\mathbb{R}^n$, and $c$ is a scalar, then
				\begin{enumerate}[a.]
					\item $A(\mathbf{u}+\mathbf{v})=A\mathbf{u}+A\mathbf{v}$;
					\item $A(c\mathbf{u})=c(A\mathbf{u})$.
				\end{enumerate}
		\subsection{Solution Set of Linear Systems}
			\subsubsection{Homogeneous Linear Systems}
				The homogeneous equation $A\mathbf{x}=\mathbf{0}$ has a nontrivial solution if and only if the equation has at least one free variable.
			\subsubsection{Theorem 6}
				Suppose the equation $A\mathbf{x}=\mathbf{b}$ is consistent for some given $\mathbf{b}$, and let $\mathbf{p}$ be a solution. Then the solution set of $A\mathbf{x}=\mathbf{b}$ is the set of all vectors of the form $\mathbf{w}=\mathbf{p}+\mathbf{v}_h$, where $\mathbf{v}_h$ is any solution of the homogeneous equation $A\mathbf{x}=\mathbf{0}$.
		\subsection{Applications of Linear Systems}
		\subsection{Linear Independence}
			\subsubsection{Definition}
				An indexed set of vectors \{$\mathbf{v}_1,\cdots,\mathbf{v}_p$\} in $\mathbb{R}^n$ is said to be \textbf{linearly independent} if the vector equation
				\begin{equation}
					x_1\mathbf{v}_1+x_2\mathbf{v}_2+\cdots+x_p\mathbf{v}_p=\mathbf{0}
				\end{equation}
				has only trivial solution. The set \{$\mathbf{v}_1$,$\cdots$,$\mathbf{v}_p$\} is said to be \textbf{linearly dependent} if there exists weights $c_1$,$\cdots$,$c_p$, not all zero, such that
				\begin{equation}
					c_1\mathbf{v}_1+c_2\mathbf{v}_2+\cdots+c_p\mathbf{v}_p=\mathbf{0}
				\end{equation}
			\subsubsection{Linear Independence of Matrix Columns}
				The columns of a matrix $A$ are linearly independent if and only if the equation $A\mathbf{x}=\mathbf{0}$ has \textit{only} trivial solution.
			\subsubsection{Sets of One or Two Vectors}
				A set of two vectors \{$\mathbf{v}_1$, $\mathbf{v}_2$\} is linearly dependent if at least one of the vectors is a multiple of the other. The set is linearly independent if and only if neither of the vectors is a multiple of each other.
			\subsubsection{Theorem 7}
				\textbf{Characterization of Linearly Dependent Sets:} An indexed set $S=\{\mathbf{v}_1, \cdots, \mathbf{v}_p\}$ of two or more vectors is linearly dependent if and only if at least one of the vectors in $S$ is a linear combination of the others. In fact, if $S$ is linearly dependent and $\mathbf{v}_1\neq\mathbf{0}$, then some $\mathbf{v}_j$ (with $j>1$) is a linear combination of the preceding vectors, $\mathbf{v}_1,\cdots,\mathbf{v}_{j-1}$.
			\subsubsection{Theorem 8}
				If a set contains more vectors than there are entries in each vector, then the set is linearly dependent. That is, any set \{$\mathbf{v}_1,\cdots,\mathbf{v}_p$\} in $\mathbb{R}^n$ is linearly dependent if $p>n$.
			\subsubsection{Theorem 9}
				If a set $S=\{\mathbf{v}_1,\cdots,\mathbf{v}_p\}$ in $\mathbb{R}^n$ contains the zero vector, then the set is linearly dependent.
		\subsection{Introduction to Linear Transformations}
			\subsubsection{Definition}
				A transformation (or mapping) T is linear if:
				\begin{enumerate}[i]
					\item $T(\mathbf{u}+\mathbf{v})=T(\mathbf{u})+T(\mathbf{v})$ for all $\mathbf{u,v}$ in the domain of $T$;
					\item $T(c\mathbf{u})=cT(\mathbf{u})$ for all $\mathbf{u}$ and all scalars $c$.
				\end{enumerate}
			\subsubsection{}	
				If $T$ is a linear transformation, then
				\begin{equation}
					T(\mathbf{0})=\mathbf{0}
				\end{equation}
				and
				\begin{equation}
					T(c\mathbf{u}+d\mathbf{v})=cT(\mathbf{u})+dT(\mathbf{v})
				\end{equation}
				for all vectors $\mathbf{u,v}$ in the domain of $T$ and all scalars $c,d$.
		\subsection{The Matrix of a Linear Transformation}
			\subsubsection{Theorem 10}
				Let $T:\mathbb{R}^n\rightarrow\mathbb{R}^m$ be a linear transformation. Then there exists a unique matrix $A$ such that
				\begin{equation}
					T(\mathbf{x})=A\mathbf{x}\textrm{ for all }\mathbf{x}\textrm{ in }\mathbb{R}^n
				\end{equation}
				In fact, $A$ is the $m\times n$ matrix whose $j$th column is the vector $T(\mathbf{e}_j)$, where $\mathbf{e}_j$ is the $j$th column of the identity matrix in $\mathbb{R}^n$:
				\begin{equation}
					A=
					\begin{bmatrix}
						T(\mathbf{e}_1) & \cdots & T(\mathbf{e}_n)
					\end{bmatrix}
				\end{equation}
			\subsubsection{Definition}
				A mapping $T:\mathbb{R}^n\rightarrow\mathbb{R}^m$ is said to be \textbf{onto} $\mathbb{R}^m$ if each $\mathbf{b}$ in $\mathbb{R}^m$ is the image of \textit{at least one} $\mathbf{x}$ in $\mathbb{R}^n$.
			\subsubsection{Definition}
				A mapping $T:\mathbb{R}^n\rightarrow\mathbb{R}^m$ is said to be \textbf{one-to-one} if each $\mathbf{b}$ in $\mathbb{R}^m$ is the image of \textit{at most one} $\mathbf{x}$ in $\mathbb{R}^n$.
			\subsubsection{Theorem 11}
				Let $T:\mathbb{R}^n\rightarrow\mathbb{R}^m$ be a linear transformation. Then $T$ is one-to-one if and only if the equation $T(\mathbf{x})=\mathbf{0}$ has only the trivial solution.
			\subsubsection{Theorem 12}
				Let $T:\mathbb{R}^n\rightarrow\mathbb{R}^m$ be a linear transformation and let A be the standard matrix for $T$. Then:
				\begin{enumerate}[i]
					\item $T$ maps $\mathbb{R}^n$ onto $\mathbb{R}^m$ if and only if the columns of A span $\mathbb{R}^m$;
					\item $T$ is one-to-one if and only if the columns of $A$ are linearly independent.
				\end{enumerate}
		\subsection{Linear Models in Business, Science, and Engineering}
	\section{Matrix Algebra}
		\subsection{Matrix Operation}
			\subsubsection{Theorem 1}
				Let $A$, $B$, and $C$ be matrices of he same size, and let $r$ and $s$ be scalars.
				\begin{enumerate}[a.]
					\item $A+B=B+A$
					\item $(A+B)+C=A+(B+C)$
					\item $A+0=A$
					\item $r(A+B)=rA+rB$
					\item $(r+s)A=rA+sA$
					\item $r(sA)=(rs)A$
				\end{enumerate}
			\subsubsection{Definition}
				If $A$ is an $m\times n$ matrix, and if $B$ is an $n\times p$ matrix with columns $\mathbf{b}_1,\cdots,\mathbf{b}_p$, then the product $AB$ is the $m\times p$ matrix whose columns are $A\mathbf{b}_1,\cdots,\mathbf{b}_p$. That is,
				\begin{equation}
					AB=A
					\begin{bmatrix}
						\mathbf{b}_1 & \mathbf{b}_2 & \cdots & \mathbf{b}_p
					\end{bmatrix}=
					\begin{bmatrix}
						A\mathbf{b}_1 & A\mathbf{b}_2 & \cdots & A\mathbf{b}_p
					\end{bmatrix}
				\end{equation}
				
				Each column of $AB$ is a linear combination of the columns of $A$ using the weights from the corresponding column of $B$.
			
			\subsubsection{Theorem 2}
				Let $A$ be an $n\times m$ matrix, and let $B$ and $C$ have sizes for which the indicated sums and products are defined.
				\begin{enumerate}[a.]
					\item $A(BC)=(AB)C$		(associative law of multiplication)
					\item $A(B+C)=AB+AC$	(left distribution law)
					\item $(B+C)A=BA+CA$	(right distribution law)
					\item $r(AB)=(rA)B=A(rB)$ for any scalar $r$
					\item $I_mA=A=AIn$		(identity for matrix multiplication)
				\end{enumerate}
				
				\textbf{Warnings:}
				\begin{enumerate}
					\item In general, $AB\neq BA$.
					\item The cancellation laws do \textit{not} hold for matrix multiplication. That is, if $AB=AC$, then it is \textit{not} true in general that $B=C$.
					\item If a product $AB$ is the zero matrix, you \textit{cannot} conclude in general that either $A=0$ or $B=0$.
				\end{enumerate}
			\subsubsection{Theorem 3}
				Let $A$ and $B$ denote matrices whose sizes are appropriate for the following sums and products.
				\begin{enumerate}[a.]
					\item $(A^T)^T=A$
					\item $(A+B)^T=A^T+B^T$
					\item For any scalar $r$, $(rA)^T=rA^T$
					\item $(AB)^T=B^TA^T$
				\end{enumerate}
				
				The transpose of a product of matrices equals the product of their transposes in the \textit{reverse} order.
		\subsection{The Inverse of a Matrix}
			\subsubsection{Theorem 4}
				Let $A=\begin{bmatrix}
					a & b \\ c & d
				\end{bmatrix}$. If $ad-bc\neq 0$, then $A$ is invertible and
				\begin{equation}
					A^{-1}=\frac{1}{ad-bc}
					\begin{bmatrix}
						d & -b \\ -c & a
					\end{bmatrix}
				\end{equation}
				If $ad-bc=0$, then $A$ is not invertible.
			\subsubsection{Theorem 5}
				If $A$ is an invertible $m\times n$ matrix, then for each $\mathbf{b}$ in $\mathbb{R}^n$, the equation $A\mathbf{x}=\mathbf{b}$ has the unique solution $\mathbf{x}=A^{-1}\mathbf{b}$.
			\subsubsection{Theorem 6}
				\begin{enumerate}[a.]
					\item If $A$ is an invertible matrix, then $A^{-1}$ is invertible and
						\begin{equation}
							(A^{-1})^{-1}=A
						\end{equation}
					\item If $A$ and $B$ are $n\times n$ invertible matrices, then so is $AB$, and the inverse of $AB$ is the product of the inverses of $A$ and $B$ in the reverse order. That is,
						\begin{equation}
							(AB)^{-1}=B^{-1}A^{-1}
						\end{equation}
					\item If $A$ is an invertible matrix, then so is $A^T$, and the inverse of $A^T$ is the transpose of $A^{-1}$. That is,
						\begin{equation}
							(A^T)^{-1}=(A^{-1})^T
						\end{equation}
				\end{enumerate}
				
				The product of $n\times n$ invertible matrices is invertible, and the inverse is the product of their inverses in the reverse order.
				
				If an elementary row operation is performed on an $m\times n$ matrix $A$, the resulting matrix can be written as $EA$, where the $m\times m$ matrix $E$ is created by performing the same row operation on $I_m$.
				
				Each elementary matrix $E$ is invertible. The inverse of $E$ is the elementary matrix of the same type that transforms $E$ back into $I$.
			\subsubsection{Theorem 7}
				An $n\times n$ matrix $A$ is invertible if and only if $A$ is row equivalent to $I_n$, and in this case, any sequence of the elementary row operations that reduces $A$ to $I_n$ also transforms $I_n$ into $A^{-1}$.
		\subsection{Characterizations of Invertible Matrices}
			\subsubsection{Theorem 8}
				\textbf{The Invertible Matrix Theorem:} Let $A$ be a square $n\times n$ matrix. Then the following statements are equivalent. That is, for a given $A$, the statements are either all true or all false.
				\begin{enumerate}[a.]
					\item $A$ is an invertible matrix.
					\item $A$ is row equivalent to the $n\times n$ identity matrix.
					\item $A$ has $n$ pivot positions.
					\item The equation $A\mathbf{x}=\mathbf{0}$ has only the trivial solution.
					\item The columns of $A$ form a linearly independent set.
					\item The linear transformation $\mathbf{x}\mapsto A\mathbf{x}$ is one-to-one.
					\item The equation $A\mathbf{x}=\mathbf{b}$ has at least one solution for each $\mathbf{b}$ in $\mathbb{R}^n$.
					\item The columns of $A$ span $\mathbb{R}^n$.
					\item The linear transformation $\mathbf{x}\mapsto A\mathbf{x}$ maps $\mathbb{R}^n$ onto $\mathbb{R}^n$.
					\item There is an $n\times n$ matrix $C$ such that $CA=I$.
					\item There is an $n\times n$ matrix $D$ such that $AD=I$.
					\item $A^T$ is an invertible matrix.
				\end{enumerate}
				
			Let $A$ and $B$ be square matrices. If $AB=I$, then $A$ and $B$ are both invertible, with $B=A^{-1}$ and $A=B^{-1}$.
			
			\subsubsection{Theorem 9}
				Let $T:\mathbb{R}^n\rightarrow\mathbb{R}^n$ be a linear transformation and let $A$ be the standard matrix for $T$. Then $T$ is invertible if and only if $A$ is an invertible matrix. In that case, the linear transformation $S$ given by $S(\mathbf{x})=A^{-1}\mathbf{x}$ is the unique function satisfying (1) and (2).
		\subsection{Partitioned Matrices}
			\subsubsection{Theorem 10}
				\textbf{Column-Row Expansion of $AB$:} If $A$ is $m\times n$ and $B$ is $n\times p$, then
				\begin{eqnarray}
					AB=\begin{bmatrix}
						\textrm{col}_1(A) & \textrm{col}_2(A) & \cdots & \textrm{col}_n(A)
					\end{bmatrix}
					\begin{bmatrix}
						\textrm{row}_1(B) \\ \textrm{row}_2(B) \\ \vdots \\ \textrm{row}_n(B)
					\end{bmatrix}\\
					=\textrm{col}_1(A)\textrm{row}_1(B)+\cdots +\textrm{col}_n(A)\textrm{row}_n(B)
				\end{eqnarray}
		\subsection{Matrix Factorizations}
		\subsection{The Leontief Input-Output Model}
		\subsection{Applications to Computer Graphics}
		\subsection{Subspace of $\mathbb{R}^n$}
			\subsubsection{Definition}
				A \textbf{subspace} of $\mathbb{R}^n$ is any set $H$ in $\mathbb{R}^n$ that has three properties:
				\begin{enumerate}[a.]
					\item The zero vector is in $H$.
					\item For each $\mathbf{u}$ and $\mathbf{v}$ in $H$, the sum $\mathbf{u}+\mathbf{v}$ is in $H$.
					\item For each $\mathbf{u}$ in $H$ and each scalar $c$, the vector $c\mathbf{u}$ is in $H$.
				\end{enumerate}
			\subsubsection{Definition}
				The \textbf{column space} of a mtrix $A$ is the set Col $A$ of all linear combinations of the columns of $A$.
			\subsubsection{Definition}
				The \textbf{null space} of a matrix $A$ is the set Nul $A$ of all solutions to the homogeneous equation $A\mathbf{x}=\mathbf{0}$.
			\subsubsection{Theorem 12}
				The null space of an $m\times n$ matrix $A$ is a subspace of $\mathbb{R}^n$. Equivalently, the set of all solutions to a system $A\mathbf{x}=\mathbf{0}$ of $m$ homogeneous linear equations in $n$ unknowns is a subspace of  $\mathbb{R}^n$.
			\subsubsection{Definition}
				A basis for a subspace $H$ of $\mathbb{R}^n$ is a linearly independent set in $H$ that spans $H$.
			\subsubsection{Theorem 13}
				The pivot columns of a matrix $A$ form a basis for the column space of $A$.
		\subsection{Dimension and Rank}
			\subsubsection{Definition}
				Suppose the set $\mathcal{B}=\{\mathbf{b}_1,\cdots,\mathbf{b}_p\}$ is a basis for a subspace $H$. For each $\mathbf{x}$ in $H$, the \textbf{coordinates of x relative to the basis} $\mathcal{B}$ are the weights $c_1,\cdots,c_p$ such that $\mathbf{x}=c_1\mathbf{b}_1+\cdots+c_p\mathbf{b}_p$, and the vector in $\mathbb{R}^p$
				\begin{equation}
					[\mathbf{x}]_{\mathcal{B}}=
					\begin{bmatrix}
						c_1 \\ \vdots \\ c_p
					\end{bmatrix}
				\end{equation}
				is called the \textbf{coordinate vector of x (relative to $\mathcal{B}$)} or the \textbf{$\mathcal{B}$-coordinate vector of x}.
			\subsubsection{Definition}
				The \textbf{dimension} of a nonzero subspace $H$, denoted by dim $H$, is the number of vectors in any basis for $H$. The dimension of the zero subspace $\{\mathbf{0}\}$ is defined to be zero.
			\subsubsection{Definition}
				The \textbf{rank} of a matrix $A$, denoted by rank $A$, is the dimension of the column space of $A$.
			\subsubsection{Theorem 14}
				\textbf{The Rank Theorem:} If a matrix $A$ has $n$ columns, then rank $A$ + dim Nul $A=n$.
			\subsubsection{Theorem 15}
				\textbf{The Basis Theorem:} Let $H$ be a p-dimensional subspace of $\mathbb{R}^n$. Any linearly independent set of exactly $p$ elements in $H$ is automatically a basis for $H$. Also, any set of $p$ elements of $H$ that spans $H$ is automatically a basis for $H$.
			\subsubsection{The Invertible Matrix Theorem (continue)}
				Let $A$ be an $n\times n$ matrix. Then the following statements are each equivalent to the statement that $A$ is an invertible matrix.
				\begin{enumerate}[a.]
					\item The columns of $A$ form a basis of $\mathbb{R}^n$.
					\item Col $A=\mathbb{R}^n$
					\item dim Col $A=n$
					\item rank $A=n$
					\item Nul $A=\{\mathbf{0}\}$
					\item dim Nul $A=0$
				\end{enumerate}
	\section{Determinants}
		\subsection{Introduction to Determinants}
			\subsubsection{Definition}
				For $n\geq 2$, the determinant of an $n\times n$ matrix $A=[a_{ij}]$ is the sum of $n$ terms of the form $\pm a_{1j}$det$A_{1j}$, with plus and minus signs alternating, where the entries $a_{11},a_{12},\cdots,a_{1n}$ are from the first row of $A$. In symbols,
				\begin{eqnarray}					\textrm{det}A=a_{11}\textrm{det}A_{11}-a_{12}\textrm{det}A_{12}+\cdots+(-1)^{1+n}a_{1n}\textrm{det}A_{1n}\\
				=\sum_{j=1}^{n}(-1)^{1+j}a_{1j}\textrm{det}A_{1j}
				\end{eqnarray}
			\subsubsection{Theorem 1}
				The determinant of an $n\times n$ matrix $A$ can be computed by a cofactor expansion across any row or down any column. The expansion across the $i$th row using the cofactors $C_{ij}=(-1)^{i+j}$det$A_{ij}$ is
				\begin{equation}
					\textrm{det}A=a_{i1}C_{i1}+a_{i2}C_{i2}+\cdots+a_{in}C_{in}
				\end{equation}
				The cofactor expansion down the $j$th column is
				\begin{equation}
					\textrm{det}A=a_{1j}C_{1j}+a_{2j}C_{2j}+\cdots+a_{nj}C_{nj}
				\end{equation}
			\subsubsection{Theorem 2}
				If $A$ is a triangular matrix, then det $A$ is the product of the entires on the main diagonal of $A$.
		\subsection{Properties of Determinants}
			\subsubsection{Theorem 3}
				\textbf{Row Operations:} Let $A$ be a square matrix.
				\begin{enumerate}[a.]
					\item If a multiple of one row of $A$ is added to another row to produce a matrix B, then $\textrm{det}B$ = $\textrm{det}A$.
					\item If two rows of $A$ are interchanged to produce $B$, then $\textrm{det}B$ = $-\textrm{det}A$.
					\item If one row of $A$ is multiplied by $k$ to produce $B$, then $\textrm{det}B$ = $k\cdot\textrm{det}A$.
				\end{enumerate}
			\subsubsection{Theorem 4}
				A square matrix $A$ is invertible if and only if $\textrm{det}A\neq 0$.
			\subsubsection{Theorem 5}
				If $A$ is an $n\times n$ matrix, then $\textrm{det}A^T=\textrm{det}A$.
			\subsubsection{Theorem 6}
				\textbf{Multiplicative Property:} If $A$ and $B$ are $n\times n$ matrices, then $\textrm{det}AB=(\textrm{det}A)(\textrm{det}B)$.
		\subsection{Cramer's Rule, Volume, and Linear Transformations}
			\subsubsection{Theorem 7}
				\textbf{Cramer's Rule:} Let $A$ be an invertible $n\times n$ matrix. For any $\mathbf{b}$ in $\mathbb{R}^n$, the unique solution $\mathbf{x}$ of $A\mathbf{x}=\mathbf{b}$ has entries given by
				\begin{equation}
					x_i=\frac{\textrm{det}A_i(\mathbf{b})}{\textrm{det}A}\textrm{, }i=1,2,\cdots,n
				\end{equation}
			\subsubsection{Theorem 8}
				\textbf{An Inverse Formula:} Let $A$ be an invertible $n\times n$ matrix. Then
				\begin{equation}
					A^{-1}=\frac{1}{\textrm{det}A}\textrm{adj}A
				\end{equation}
			\subsubsection{Theorem 9}
				If $A$ is a $2\times 2$ matrix, the area of the parallelogram determined by the columns of $A$ is $|\textrm{det}A|$. If $A$ is a $3\times 3$ matrix, the volume of the parallelepiped determined by the columns of $A$ is $|\textrm{det}A|$.
			\subsubsection{Theorem 10}
				Let $T:\mathbb{R}^2\rightarrow\mathbb{R}^2$ be the linear transformation determined by a $2\times 2$ matrix $A$. If $S$ is a parallelogram in $\mathbb{R}^2$, then
				\begin{equation}
					\{\textrm{area of }T(S)\}=|\textrm{det}A|\cdot\{\textrm{area of }S\}
				\end{equation}
				If $T$ is determined by a $3\times 3$ matrix $A$, and if $S$ is a parallelepiped in $\mathbb{R}^3$, then
				\begin{equation}
					\{\textrm{volume of }T(S)\}=|\textrm{det}A|\cdot\{\textrm{volume of }S\}
				\end{equation}
	\section{Vector Spaces}
		\subsection{Vector Spaces and Subspaces}
			\subsubsection{Definition}
				A \textbf{vector space} is a nonempty set $V$ of objects, called \textit{vectors}, on which are defined two operations, called \textit{addition} and \textit{multiplication by scalars} (real numbers), subject to the ten axioms (or rules) listed below. The axioms must hold for all vectors $\mathbf{u}$, $\mathbf{v}$, and $\mathbf{w}$ in $V$ for all scalars $c$ and $d$.
				\begin{enumerate}
					\item The sum of $\mathbf{u}$ and $\mathbf{u}$, denoted $\mathbf{u}+\mathbf{v}$, is in $V$.
					\item $\mathbf{u}+\mathbf{v}=\mathbf{v}+\mathbf{u}$
					\item $(\mathbf{u}+\mathbf{v})+\mathbf{w}=\mathbf{u}+(\mathbf{v}+\mathbf{w})$
					\item There is a \textbf{zero} vector $\mathbf{0}$ in $V$ such that $\mathbf{u}+\mathbf{0}=\mathbf{u}$.
					\item For each $\mathbf{u}$ in $V$, there is a vector $-\mathbf{u}$ in $V$ such that $\mathbf{u}+(-\mathbf{u})=\mathbf{0}$.
					\item The scalar multiple of $\mathbf{u}$ by $c$, denoted by $c\mathbf{u}$, is in $V$.
					\item $c(\mathbf{u}+\mathbf{v})=c\mathbf{u}+c\mathbf{v}$
					\item $(c+d)\mathbf{u}=c\mathbf{u}+d\mathbf{u}$
					\item $c(d\mathbf{u})=(cd)\mathbf{u}$
					\item $1\mathbf{u}=\mathbf{u}$
				\end{enumerate}
			\subsubsection{Definition}
				A \textbf{subspace} of a vector space $V$ is a subset $H$ of $V$ that has three properties:
				\begin{enumerate}[a.]
					\item The zero vector of $V$ is in $H$.
					\item $H$ is closed under vector addition. That is, for each $\mathbf{u}$ and $\mathbf{v}$ in $H$, the sum $\mathbf{u}+\mathbf{v}$ is in $H$.
					\item $H$ is closed under multiplication by scalars. That is, for each $\mathbf{u}$ in $H$ and each scalar $c$, the vector $c\mathbf{u}$ is in $H$.
				\end{enumerate}
			\subsubsection{Theorem 1}
				If $\mathbf{v}_1,\cdots,\mathbf{v}_p$ are in a vector space $V$, then Span $\{\mathbf{v}_1,\cdots,\mathbf{v}_p\}$ is a subspace of $V$.
		\subsection{Null Spaces, Column Spaces, and Linear Transformation}
			\subsubsection{Definition}
				The \textbf{null space} of an $m\times n$ matrix $A$, written as Nul $A$, is the set of all solutions to the homogeneous equation $A\mathbf{x}=\mathbf{0}$. In set notation,
				\begin{equation}
					\textrm{Nul}A=\{\mathbf{x}:\mathbf{x}\textrm{ is in }\mathbb{R}^n\textrm{ and }A\mathbf{x}=\mathbf{0}\}
				\end{equation}
			\subsubsection{Theorem 2}
				The null space of an $m\times n$ matrix $A$ is a subspace of $\mathbb{R}^n$. Equivalently, the set of all solutions to a system $A\mathbf{x}=\mathbf{0}$ of $m$ homogeneous linear equations in $n$ unknowns is a subspace of $\mathbb{R}^n$.
			\subsubsection{Definition}
				The column space of an $m\times n$ matrix $A$, is the set of all linear combinations of the columns of $A$. If $A=\begin{bmatrix}
					\mathbf{a}_1 & \cdots & \mathbf{a}_n
				\end{bmatrix}$, then
				\begin{equation}
					\textrm{Col }A=\textrm{Span}\{\mathbf{a}_1,\cdots,\mathbf{a}_n\}
				\end{equation}
			\subsubsection{Theorem 3}
				The column space of an $m\times n$ matrix $A$ is a subspace of $\mathbb{R}^m$.
				\begin{equation}
					\textrm{Col }A=\{\mathbf{b}:\mathbf{b}=A\mathbf{x}\textrm{ for some }\mathbf{x}\textrm{ in }\mathbb{R}^n\}
				\end{equation}
				
				The column space of an $m\times n$ matrix $A$ is all of $\mathbb{R}^m$ if and only if the equation $A\mathbf{x}=\mathbf{b}$ has a solution for each $\mathbf{b}$ in $\mathbb{R}^m$.
			\subsubsection{Definition}
				A \textbf{linear transform} $T$ from a vector space $V$ into a vector space $W$ is a rule that assigns to each vector $\mathbf{x}$ in $V$ a unique vector $T(\mathbf{x})$ in $W$, such that
				\begin{enumerate}[(i)]
					\item $T(\mathbf{u}+\mathbf{v})=T(\mathbf{u})+T(\mathbf{v})$ for all $\mathbf{u,v}$ in $V$, and
					\item $T(c\mathbf{u})=cT(\mathbf{u})$ for all $\mathbf{u}$ in $V$ and all scalars $c$.
				\end{enumerate}
		\subsection{Linearly Independent Sets; Bases}
			\subsubsection{Theorem 4}
				An indexed set $\{\mathbf{v}_1,\dots,\mathbf{v}_p\}$ of two or more vectors, with $\mathbf{v_1}\neq\mathbf{0}$, is linearly dependent if and only if some $\mathbf{v}_j$ (with $j>1$) is a linear combination of the preceding vectors, $\mathbf{v}_1,\dots,\mathbf{v}_{j-1}$.
			\subsubsection{Definition}
				Let $H$ be a subspace of a vector space $V$. An indexed set of vectors $\mathcal{B}=\{\mathbf{b}_1,\dots,\mathbf{b}_p\}$ in $V$ is a \textbf{basis} for $H$ if
				\begin{enumerate}[(i)]
					\item $\mathcal{B}$ is a linearly independent set, and
					\item the subspace spanned by $\mathcal{B}$ coincides with $H$; that is,
						\begin{equation}
							H=\textrm{Span}\{\mathbf{b}_1,\dots,\mathbf{b}_p\}
						\end{equation}
				\end{enumerate}
			\subsubsection{Theorem 5}
				\textbf{The Spanning Set Theorem:} Let $S=\{\mathbf{v}_1,\dots,\mathbf{v}_p\}$ be a set in $V$, and let $H=\{\mathbf{v}_1,\dots,\mathbf{v}_p\}$.
				\begin{enumerate}[a.]
					\item If one of the vectors in $S$--say, $\mathbf{v}_k$--is a linear combination of the remaining vectors in $S$, then the set formed from $S$ by removing $\mathbf{v}_k$ still spans $H$.
					\item If $H\neq\{\mathbf{0}\}$, some subset of $S$ is a basis for $H$.
				\end{enumerate}
				
				Elementary row operations on a matrix do not affect the linear dependence relations among the columns of the matrix.
			\subsubsection{Theorem 6}
				The pivot columns of a matrix $A$ form a basis for Col $A$.
		\subsection{Coordinate Systems}
			\subsubsection{Theorem 7}
				\textbf{The Unique Representation Theorem:} Let $\mathcal{B}=\{\mathbf{b}_1,\dots,\mathbf{b}_n\}$ be a basis for a vector space $V$. Then for each $\mathbf{x}$ in $V$, there exists a unique set of scalars $c_1,\dots,c_n$ such that
				\begin{equation}
					\mathbf{x}=c_1\mathbf{b}_1+\cdots+c_n\mathbf{b}_n
				\end{equation}
			\subsubsection{Definition}
				Suppose $\mathcal{B}=\{\mathbf{b}_1,\dots,\mathbf{b}_n\}$ is a basis for $V$ and $\mathbf{x}$ is in $V$.The \textbf{coordinates of x relative to the basis $\mathcal{B}$} (or the \textbf{$\mathcal{B}$-coordinates of x}) are the weights $c_1,\dots,c_n$ such that $\mathbf{x}=c_1\mathbf{b}_1+\cdots+c_n\mathbf{b}_n$
			\subsubsection{Theorem 8}
				Let $\mathcal{B}=\{\mathbf{b}_1,\dots,\mathbf{b}_n\}$ be a basis for a vector space $V$. Then the coordinate mapping $\mathbf{x}\mapsto[\mathbf{x}]_\mathcal{B}$ is a one-to-one linear transformation from $V$ onto $\mathbb{R}^n$.
		\subsection{The Dimension of a Vector Space}
			\subsubsection{Theorem 9}
				If a vector space $V$ has a basis $\mathcal{B}=\{\mathbf{b}_1,\dots,\mathbf{b}_n\}$, then any set in $V$ containing more than $n$ vectors must be linearly dependent.
			\subsubsection{Theorem 10}
				If a vector space $V$ has a basis of $n$ vectors, then every basis of $V$ must consist of exactly $n$ vectors.
			\subsubsection{Definition}
				If $V$ is spanned by a finite set, then $V$ is said to be \textbf{finite-dimensional}, and the dimension of $V$, written as dim $V$, is the number of vectors in a basis for $V$. The dimension of the zero vector space $\{\mathbf{0}\}$ is defined to be zero. If $V$ is not spanned by a finite set, then $V$ is said to be \textbf{infinite-dimensional}.
			\subsubsection{Theorem 11}
				Let $H$ be a space of a finite-dimensional vector space $V$. Any linearly independent set in $H$ can be expanded, if necessary, to a basis for $H$. Also, $H$ is finite-dimensional and
				\begin{equation}
					\textrm{dim} H\leq\textrm{dim} V
				\end{equation}
			\subsubsection{Theorem 12}
				\textbf{The Basis Theorem:} Let $V$ be a $p$-dimensional vector space, $p\geq 1$. Any linearly independent set of exactly $p$ elements in $V$ is automatically a basis for $V$. Any set of exactly $p$ elements that spans $V$ is automatically a basis for $V$.
				
				The dimension of Nul $A$ is the number of free variables in the equation $A\mathbf{x}=\mathbf{0}$, and the dimension of Col $A$ is the number of pivot columns in $A$.
		\subsection{Rank}
			\subsubsection{Theorem 13}
				If two matrices $A$ and $B$ are row equivalent, then their row spaces are the same. If $B$ is in echelon form, the nonzero rows of $B$ form a basis for the row space of $A$ as well as for that of $B$.
			\subsubsection{Definition}
				The \textbf{rank} of $A$ is the dimension of the column space of $A$.
			\subsubsection{Theorem 14}
				\textbf{The Rank Theorem:} The dimensions of the column space and the row space of an $m\times n$ matrix $A$ are equal. This common dimension, the rank of $A$, also equals the number of pivot positions in $A$ and satisfies the equation
				\begin{equation}
					\textrm{rank }A+\textrm{dim Nul }A=n
				\end{equation}
			\subsubsection{Theorem}
				\textbf{The Invertible Matrix Theorem (continued):} Let $A$ be an $n\times n$ matrix. Then the following statements are each equivalent to the statement that $A$ is an invertible matrix.
				\begin{enumerate}[a.]
					\item The columns of $A$ form a basis of $\mathbb{R}^n$.
					\item Col $A=\mathbb{R}^n$
					\item dim Col $A=n$
					\item rank $A=n$
					\item Nul $A=\{\mathbf{0}\}$
					\item dim Nul $A=0$
				\end{enumerate}
		\subsection{Change of Basis}
			\subsubsection{Theorem 15}
				Let $\mathcal{B}=\{\mathbf{b}_1,\dots,\mathbf{b}_n\}$ and $\mathcal{C}=\{\mathbf{c}_1,\dots,\mathbf{c}_n\}$ be bases of a vector space $V$. Then there is a unique $n\times n$ matrix $\underset{\mathcal{C}\longleftarrow\mathcal{B}}{P}$ such that
				\begin{equation}
					[\mathbf{x}]_\mathcal{C}=\underset{\mathcal{C}\longleftarrow\mathcal{B}}{P}[\mathbf{x}]_\mathcal{C}
				\end{equation}
				The columns of $\underset{\mathcal{C}\longleftarrow\mathcal{B}}{P}$ are the $\mathcal{C}$-coordinate vectors of the vectors in the basis $\mathcal{B}$. That is,
				\begin{equation}
					\underset{\mathcal{C}\longleftarrow\mathcal{B}}{P}=\begin{bmatrix}
						[\mathbf{b}_1]_\mathcal{C} & [\mathbf{b}_c]_\mathcal{C} & \cdots & [\mathbf{b}_n]_\mathcal{C}
					\end{bmatrix}
				\end{equation}
		\subsection{Application to Difference Equations}
			\subsubsection{Theorem 16}
				If $a_n\neq 0$ and if $\{z_k\}$ is given, the equation
				\begin{equation}
					y_{k+n}+a_1y_{k+n-1}+\cdots+a_{n-1}y_{k+1}+a_ny_k=z_k\textrm{  for all }k
				\end{equation}
				has a unique solution whenever $y_0,\cdots,y_{n-1}$ are specified.
			\subsubsection{Theorem 17}
				The set $H$ of all solutions of the $n$th-orderr homogeneous linear difference equation
				\begin{equation}
				y_{k+n}+a_1y_{k+n-1}+\cdots+a_{n-1}y_{k+1}+a_ny_k=z_k\textrm{  for all }k
				\end{equation}
				is an $n$-dimensional vector space.
		\subsection{Application to Markov Chains}
			\subsubsection{Theorem 18}
				If $P$ is an $n\times n$ regular stochastic matrix, then $P$ has a unique steady-state vector $\mathbf{q}$. Further, if $\mathbf{x}_0$ is any initial state and $\mathbf{x}_{k+1}=P\mathbf{x}_k$ for $k=0,1,2,\dots$, then the Markov chain $\{\mathbf{x}_k\}$ converges to $\mathbf{q}$ as $k\rightarrow\infty$.
	\section{Eigenvalues and Eigenvectors}
		\subsection{Eigenvectors and Eigenvalues}
			\subsubsection{Definition}
				An \textbf{eigenvector} of an $n\times n$ matrix $A$ is a nonzero vector $\mathbf{x}$ such that $A\mathbf{x}=\lambda\mathbf{x}$ for some scalar $\lambda$. A scalar $\lambda$ is called an \textbf{eigenvalue} of $A$ if there is a nontrivial solution $\mathbf{x}$ of $A\mathbf{x}=\lambda\mathbf{x}$; such an $\mathbf{x}$ is called an \textit{eigenvector corresponding to $\lambda$}.
			\subsubsection{Theorem 1}
				The eigenvalue of a triangular matrix are the entries on its main diagonal.
			\subsubsection{Theorem 2}
				If $\mathbf{v}_1,\dots,\mathbf{v}_r$ are eigenvectors that correspond to distinct eigenvalues $\lambda_1,\dots,\lambda_r$ of an $n\times n$ matrix $A$, then the set $\{\mathbf{v}_1,\dots,\mathbf{v}_r\}$ is linearly independent.
		\subsection{The Characteristic Equation}
			\subsubsection{Theorem}
				\textbf{The Invertible Matrix Theorem: (continued)} Let $A$ be an $n\times n$ matrix. Then $A$ is invertible if and only if:
				\begin{enumerate}[a.]
					\item The number 0 is \textit{not} an eigenvalue of $A$.
					\item The determinant of $A$ is \textit{not} zero.
				\end{enumerate}
			\subsubsection{Theorem 3}
				\textbf{Property of Determinants:} Let $A$ and $B$ be $n\times n$ matrices.
				\begin{enumerate}[a.]
					\item $A$ is invertible if and only if det $A\neq 0$.
					\item det $AB=(\textrm{det }A)(\textrm{det }B)$.
					\item det $A^T=$det $A$.
					\item If $A$ triangular, then det $A$ is the product of the entries on the main diagonal of $A$.
					\item A row replacement operation on $A$ does not change the determinant. A row interchange changes the sign of the determinant. A row scaling also scales the determinant by the same scalar factor.
				\end{enumerate}
			\subsubsection{The Characteristic Equation}
				A scalar $\lambda$ is an eigenvalue of an $n\times n$ matrix $A$ if and only if $\lambda$ satisfies the characteristic equation
				\begin{equation}
					\textrm{det }(A-\lambda I)=0
				\end{equation}
			\subsubsection{Theorem 4}
				If $n\times n$ matrices $A$ and $B$ are similar, then they have the same characteristic polynomial and hence the same eigenvalues (with the same multiplicities).
		\subsection{Diagonalization}
			\subsubsection{Theorem 5}
				\textbf{The Diagonalization Theorem:} An $n\times n$ matrix $A$ is diagonalizable if and only if $A$ has $n$ linearly independent eigenvectors.
				
				In fact, $A=PDP^{-1}$, with $D$ a diagonal matrix, if and only if the columns of $P$ are $n$ linearly independent eigenvectors of $A$. In this case, the diagonal entries of $D$ are eigenvalues of $A$ that correspond, respectively, to the eigenvectors in $P$.
			\subsubsection{Theorem 6}
				An $n\times n$ matrix with $n$ distinct eigenvalues is diagonalizable.
			\subsubsection{Theorem 7}
				Let $A$ be an $n\times n$ matrix whose distinct eigenvalues are $\lambda_1,\dots,\lambda_p$.
				\begin{enumerate}[a.]
					\item For $1\leq k\leq p$, the dimension of the eigenspace for $\lambda_k$ is less than or equal to the multiplicity of the eigenvalue $\lambda_k$.
					\item The matrix $A$ is diagonalizable if and only if the sum of the dimensions of the distinct eigenspaces equals to $n$, and this happens if and only if the dimension of the eigenspace for each $\lambda_k$ equals the multiplicity of $\lambda_k$.
					\item If $A$ is diagonalizable and $\mathcal{B}_k$ is a basis for the eigenspace corresponding to $\lambda_k$ for each $k$, then the total collection of vectors in the sets $\mathcal{B}_1,\dots,\mathcal{B}_p$ forms an eigenvector basis for $\mathbb{R}^n$.
				\end{enumerate}
		\subsection{Eigenvectors and Linear Transformations}
			\subsubsection{Theorem 8}
				\textbf{Diagonal Matrix Representation:} Suppose $A=PDP^{-1}$, where $D$ is a diagonal $n\times n$ matrix. If $\mathcal{B}$ is the basis for $\mathbb{R}^n$ formed from the columns of $P$, then $D$ is the $\mathcal{B}$-matrix for the transformation $\mathbf{x}\mapsto A\mathbf{x}$.
		\subsection{Complex Eigenvalues}
			\subsubsection{Theorem 9}
				Let $A$ be a real $2\times 2$ matrix with a complex eigenvalue $\lambda=a-bi(b\neq 0)$ and an associated eigenvector $\mathbf{v}$ in $\mathbb{C}^2$. Then
				\begin{equation}
					A=PCP^{-1}\textrm{, where }P=\begin{bmatrix}
					\textrm{Re }\mathbf{v} & \textrm{Im }\mathbf{v}
					\end{bmatrix}\textrm{ and }C=\begin{bmatrix}
					a & -b \\ b & a
					\end{bmatrix}
				\end{equation}
		\subsection{Discrete Dynamical Systems}
		\subsection{Applications to Differential Equations}
		\subsection{Iterative Estimates for Eigenvalues}
	\section{Orthogonality and Least Squares}
		\subsection{Inner Product, Length, and Orthogonality}
			\subsubsection{Theorem 1}
				Let $\mathbf{u}$, $\mathbf{v}$, and $\mathbf{w}$ be vectors in $\mathbb{R}^n$, and let $c$ be a scalar. Then
				\begin{enumerate}[a.]
					\item $\mathbf{u\cdot v}=\mathbf{v\cdot u}$
					\item $(\mathbf{u+v})\cdot\mathbf{w}=\mathbf{u}\cdot\mathbf{w}+\mathbf{v}\cdot\mathbf{w}$
					\item $(c\mathbf{u})\cdot\mathbf{v}=c(\mathbf{u}\cdot\mathbf{v})=\mathbf{u}\cdot(c\mathbf{v})$
					\item $\mathbf{u}\cdot\mathbf{u}\geq 0$, and $\mathbf{u}\cdot\mathbf{u}=0$ if and only if $\mathbf{u}=\mathbf{0}$
				\end{enumerate}
			\subsubsection{Definition}
				The \textbf{length} (or \textbf{norm}) of $\mathbf{v}$ is the nonnegative scalar $||\mathbf{v}||$ defined by
				\begin{equation}
					||\mathbf{v}||=\sqrt{\mathbf{v\cdot v}}=\sqrt{v^2_1+v^2_2+\cdots+v^2_n}\textrm{ and }||\mathbf{v}||^2=\mathbf{v\cdot v}
				\end{equation}
			\subsubsection{Definition}
				For $\mathbf{u}$ and $\mathbf{v}$ in $\mathbb{R}^n$, the \textbf{distance between u and v}, written as dist($\mathbf{u,v}$), is the length of the vector $\mathbf{u}-\mathbf{v}$. That is,
				\begin{equation}
					\mathrm{dist}(\mathbf{u,v})=||\mathbf{u}-\mathbf{v}||
				\end{equation}
			\subsubsection{Definition}
				Two vectors $\mathbf{u}$ and $\mathbf{v}$ in $\mathbb{R}^n$ are \textbf{orthogonal} (to each other) if $\mathbf{u\cdot v}=0$.
			\subsubsection{Theorem 2}
				\textbf{The Pythagorean Theorem:} Two vectors $\mathbf{u}$ and $\mathbf{v}$ are orthogonal if and only if $||\mathbf{u}+\mathbf{v}||^2=||\mathbf{u}||^2+||\mathbf{v}||^2$.
			\subsubsection{Theorem 3}
				Let $A$ be an $m\times n$ matrix. The orthogonal complement of the row space of $A$ is the nullspace of $A$, and the orthogonal complement of the column space of $A$ is the nullspace of $A^T$:
				\begin{equation}
					(\textrm{Row }A)^\bot=\textrm{Nul }A\textrm{ and }(\textrm{Col }A)^\bot=\textrm{Nul }A^T
				\end{equation}
		\subsection{Orthogonal Sets}
			\subsubsection{Theorem 4}
				If $S=\{\mathbf{u}_1,\dots,\mathbf{u}_p\}$ is an orthogonal set of nonzero vectors in $\mathbb{R}^n$, then $S$ is linearly independent and hence is a basis for the subspace spanned by $S$.
			\subsubsection{Definition}
				An \textbf{orthogonal basis} for a subspace $W$ of $\mathbb{R}^n$ is a basis for $W$ that is also an orthogonal set.
			\subsubsection{Theorem 5}
				Let $\{\mathbf{u}_1,\dots,\mathbf{u}_p\}$ be an orthogonal basis for a subspace $W$ of $\mathbb{R}^n$. For each $\mathbf{y}$ in $W$, the weights in the linear combination
				\begin{equation}
					\mathbf{y}=c_1\mathbf{u}_1+\cdots+c_p\mathbf{u}_p
				\end{equation}
				are given by
				\begin{equation}
					c_j=\frac{\mathbf{y}\cdot\mathbf{u}_j}{\mathbf{u}_j\cdot\mathbf{u}_j}\textrm{   }(j=1,\dots,p)
				\end{equation}
			\subsubsection{Theorem 6}
				An $m\times n$ matrix $U$ has orthonormal columns if and only if $U^TU=I$.
			\subsubsection{Theorem 7}
				Let $U$ be an $m\times n$ matrix with orthonormal columns, and let $\mathbf{x}$ and $\mathbf{y}$ be in $\mathbb{R}^n$. Then
				\begin{enumerate}[a.]
					\item $||U\mathbf{x}||=||\mathbf{x}||$
					\item $(U\mathbf{x})\cdot(U\mathbf{y})=\mathbf{x}\cdot\mathbf{y}$
					\item $(U\mathbf{x})\cdot(U\mathbf{y})=0$ if and only if $\mathbf{x}\cdot\mathbf{y}=0$
				\end{enumerate}
		\subsection{Orthogonal Projections}
			\subsubsection{Theorem 8}
				\textbf{The Orthogonal Decomposition Theorem:} Let $W$ be a subspace of $\mathbb{R}^n$. Then each $\mathbf{y}$ in $\mathbb{R}^n$ can be written uniquely in the form
				\begin{equation}
					\mathbf{y=\hat{y}+z}
				\end{equation}
				where $\mathbf{\hat{y}}$ is in $W$ and $\mathbf{z}$ is in $W^\bot$. In fact, if $\{\mathbf{\mathbf{u}_1,\dots,\mathbf{u}_p}\}$ is any orthogonal basis of $W$, then
				\begin{equation}
					\mathbf{\hat{y}=\frac{y\cdot u_1}{u_1\cdot u_1}u_1+\cdots+\frac{y\cdot u_p}{u_p\cdot u_p}u_p}
				\end{equation}
			\subsubsection{Theorem 9}
				\textbf{The Best Approximation Theorem:} Let $W$ be a subspace of $\mathbb{R}^n$, $\mathbf{y}$ any vector in $\mathbb{R}^n$, and $\mathbf{\hat{y}}$ the orthogonal projection of $\mathbf{y}$ onto $W$ to $\mathbf{y}$, in the sense that
				\begin{equation}
					\mathbf{||y-\hat{y}||<||y-v||}
				\end{equation}
				for all $\mathbf{v}$ in $W$ distinct from $\mathbf{\hat{y}}$.
		\subsection{The Gram-Schmidt Process}
		\subsection{Least-Squares Problems}
		\subsection{Applications to Linear Models}
		\subsection{Inner Product Spaces}
			\subsubsection{Definition}
				An \textbf{inner product} on a vector space $V$ is a function that, to each pair of vectors $\mathbf{u}$ and $\mathbf{v}$ in $V$, associates a real number $\langle\mathbf{u,v}\rangle$ and satisfies the following axioms, for all $\mathbf{u}$,$\mathbf{v}$,$\mathbf{w}$ in $V$ and all scalar $c$:
				\begin{enumerate}
					\item $\mathbf{\langle\mathbf{u,v}\rangle=\langle\mathbf{v},\mathbf{u}\rangle}$
					\item $\mathbf{\langle\mathbf{u}+\mathbf{v},\mathbf{w}\rangle=\langle\mathbf{u,w}\rangle+\langle\mathbf{v},\mathbf{w}\rangle}$
					\item $\mathbf{\langle c\mathbf{u,v}\rangle=c\langle\mathbf{u,v}\rangle}$
					\item $\mathbf{\langle\mathbf{u,v}\rangle\geq 0}$ and $\langle\mathbf{u,u}\rangle=0$ if and only if $\mathbf{u=0}$
				\end{enumerate}
				A vector space with an inner product is called an \textbf{inner product space}.
			\subsubsection{The Cauchy-Schwarz Inequality}
				For all $\mathbf{u,v}$ in $V$,
				\begin{equation}
					\mathbf{|\langle u,v\rangle|\leq ||u||\textrm{ }||v||}
				\end{equation}
			\subsubsection{The Triangle Inequality}
				For all $\mathbf{u,v}$ in $V$,
				\begin{equation}
					\mathbf{||u+v||\leq ||u||+||v||}
				\end{equation}
		\subsection{Applications of Inner Product Spaces}
	\section{Symmetric Matrices and Quadratic Forms}
		\subsection{Diagonalization of Symmetric Matrices}
			\subsubsection{Theorem 1}
				If $A$ is symmetric, then any two eigenvectors from different eigenspaces are orthogonal.
			\subsubsection{Theorem 2}
				An $n\times n$ matrix $A$ is orthogonally diagonalizable if and only if $A$ is a symmetric matrix.
			\subsubsection{Theorem 3}
				\textbf{The Spectral Theorem for Symmetric Matrices:} An $n\times n$ symmetric matrix $A$ has the following properties:
				\begin{enumerate}
					\item $A$ has $n$ real eigenvalues, counting multiplicities.
					\item The dimension of the eigenspace for each eigenvalue $\lambda$ equals the multiplicity of $\lambda$ as a root of the characteristic equation.
					\item The eigenspaces are mutually orthogonal, in the sense that eigenvectors corresponding to different eigenvalues are orthogonal.
					\item $A$ is orthogonally diagonalizable.
				\end{enumerate}
		\subsection{Quadratic Forms}
			\subsubsection{Theorem 4}
				\textbf{The Principle Axes Theorem:} Let $A$ be an $n\times n$ symmetric matrix. Then there is an orthogonal change of variable, $\mathbf{x}=P\mathbf{y}$, that transforms the quadratic form $\mathbf{x}^TA\mathbf{x}$ into a quadratic form $\mathbf{y}^TD\mathbf{y}$ with no cross-product term.
\end{document}